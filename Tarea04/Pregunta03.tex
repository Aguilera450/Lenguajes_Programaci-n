\textbf{\textcolor{blue}{3.}} \Large
Aplica $\beta$-reducciones a las siguientes expresiones
para llegar a una Forma Normal, en caso de que no se pueda justifica. Además indica
en cada paso el reducto y el redex.\\

$l\;=_{def} \lambda a.a$\\
$K\;=_{def} \lambda a.\lambda b.a$\\
$S\;=_{def} \lambda a.\lambda b.\lambda a.ac(bc)$\\
$\Omega\; =_{def} (\lambda a.aa) (\lambda a.aa)$

\begin{enumerate}[a)]
%%%%%%%%%%%%%%%%%%%%%%%%      Inciso a    %%%%%%%%%%%%%%%%%%%%%%%%%%%%
    \item $\lambda a.aK\Omega$

    \textbf{Solución:}
    \begin{eqnarray*}
        \lambda_a .aK\Omega \equiv \lambda_a .a (\lambda_a .a)((\lambda a.aa) (\lambda a.aa))
        \rightarrow_{\beta}
    \end{eqnarray*}
%%%%%%%%%%%%%%%%%%%%%%%%      Inciso b    %%%%%%%%%%%%%%%%%%%%%%%%%%%%
    \item $(\lambda a.a(ll))c$
%%%%%%%%%%%%%%%%%%%%%%%%      Inciso c    %%%%%%%%%%%%%%%%%%%%%%%%%%%%
    \item $(\lambda d.\lambda e.(\lambda f.f(\lambda a.ad))e)b(\lambda c.\lambda b.cb)$
\end{enumerate}
