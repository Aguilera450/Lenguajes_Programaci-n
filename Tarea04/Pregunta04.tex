\textbf{\textcolor{blue}{4.}}\Large
Realiza la representación de los booleanos en el cálculo
$\lambda$ según la representación de los Numerales de Church.
\begin{enumerate}[a)]
%%%%%%%%%%%%%%%%%%%%%%%%      Inciso a    %%%%%%%%%%%%%%%%%%%%%%%%%%%%
    \item Define la función disyunción $\leftrightarrow$ (equivalencia) sobre los boolenos.\\
    Sabemos que a partir de las leyes de equivalencia de la lógica proposicional, tenemos que:
    \begin{equation*}
        a \leftrightarrow b \equiv (a \rightarrow b) \land (b \rightarrow a)\quad \quad \quad \quad
            a \rightarrow b \equiv \neg a \lor b \quad \quad \quad
            b \rightarrow a \equiv \neg b \lor a
        \end{equation*}
        Por lo que finalmente tenemos que:
        $a \leftrightarrow b \equiv (\neg a \lor b) \land (\neg b \lor a )$.\\

        -- Como en clase se vio que $and$ queda definido de la siguiente forma:\\
        $\land \; =_{def} \lambda a. \lambda b.((ab)F)$.\\
        Definimos las siguientes funciones como:\\
        $\lor \; =_{def} \lambda a.\lambda b. ((aT)b)$.\\
        $\neg \; =_{def} \lambda a.aFT$.\\
        $\rightarrow \; =_{def}(\lambda a.\lambda b.\lor (\neg a)b)$.\\
        $\leftrightarrow \;
        =_{def} (\lambda a. \lambda b. \land(\rightarrow ab)(\rightarrow ba))$
        \newline
        
        A continuación se muestra  una ejecución donde tomamos los valores $T$ y $T$. Esto es,
        \begin{eqnarray*}
                (((\lambda a. \lambda b. \land(\rightarrow ab)(\rightarrow ba)) T) T)
                &\rightarrow_{\beta}&
                (( \lambda b. \land(\rightarrow Tb)(\rightarrow bT)) T) \\
                &\rightarrow_{\beta}&
                ( \land(\rightarrow TT)(\rightarrow TT)) \\
                &\rightarrow_{\beta}&
                (\land((\lambda a.\lambda b.\lor (\neg a)b) TT)(\rightarrow TT))\\
                &\rightarrow_{\beta}&
                (\land((\lambda b.\lor (\neg T)b) T)(\rightarrow TT))\\
                &\rightarrow_{\beta}&
                (\land(\lor (\neg T)T)((\lambda a.\lambda b.\lor (\neg a)b) TT))\\
                &\rightarrow_{\beta}&
                (\land(\lor (\neg T)T)((\lambda b.\lor (\neg T)b) T))\\
                &\rightarrow_{\beta}&
                (\land(\lor (\neg T)T)(\lor (\neg T)T))\\
                &\rightarrow_{\beta}&
                (\land(\lor ((\lambda a.aFT) T)T)(\lor (\neg T)T))\\
                &\rightarrow_{\beta}&
                (\land(\lor (TFT)T)(\lor (\neg T)T))\\
                &\rightarrow_{\beta}&
                (\land(\lor ((TFT)T)(\lor ((\lambda a.aFT) T)T))\\
                &\rightarrow_{\beta}&
                (\land(\lor ((TFT)T)(\lor ((TFT)T))\\
                &\rightarrow_{\beta}&
                (\land((\lambda a.\lambda b. ((aT)b)) (TFT)T)\\
                & & (\lor ((TFT)T))\\
                &\rightarrow_{\beta}&
                (\land((\lambda b. ((TT)b)) ((FT)T)\\
                & & (\lor ((TFT)T))\\
                &\rightarrow_{\beta}&
                (\land \underbrace{(((TT)F) ((T)T)}_{\alpha}\\
                & & \underbrace{(\lor ((TFT)T))}_{\text{Eventualmente } \alpha}\\
                &\rightarrow_{\beta}&
                (\land(((TT)F) ((T)T)(((TT)F) ((T)T))))\\
                &\rightarrow_{\beta}&
                (\lambda a. \lambda b.((ab)F) (((TT)F) ((T)T)(((TT)F) ((T)T))))\\
                &\rightarrow_{\beta}&
                (\lambda b.((Tb)F) (((T)F) ((T)T)(((TT)F) ((T)T))))\\
                &\rightarrow_{\beta}&
                (((TT)F) ((F) ((T)T)(((TT)F) ((T)T))))\\
                &\rightarrow_{\beta}& (((\lambda_a .(\lambda_b .a)T)F) ((F) ((T)T)(((TT)F) ((T)T))))\\
                &\rightarrow_{\beta}& (((\lambda_b .T)F) ((F) ((T)T)(((TT)F) ((T)T))))\\
                &\rightarrow_{\beta}& T
        \end{eqnarray*}
        lo cual es cierto ($T \leftrightarrow T$ es $T$).
%%%%%%%%%%%%%%%%%%%%%%%%      Inciso b    %%%%%%%%%%%%%%%%%%%%%%%%%%%%
    \item Define la función $xor$ (disyunción exclusiva) sobre los booleanos.
    \begin{equation*}
            xor =_{def} \lambda a. \lambda b.((a \; (\text{not } b)) \; b))
    \end{equation*}
    A continuación se muestra  una ejecución donde tomamos los valores $T$ y $T$. Esto es,
    \begin{eqnarray*}
        (((\lambda a. \lambda b.((a \; (\text{not } b)) \; b)))T)T) &\rightarrow_{\beta}&
        ((\lambda b.((T \; (\text{not } b)) \; b)))T)\\
        &\rightarrow_{\beta}&
        (T \; (\text{not } T)) \; T)\\
        &\rightarrow_{\beta}&
        (T \; (\text{not } T)) \; T)\\
        &\rightarrow_{\beta}&
        ((\lambda_a .(\lambda_b .a)) \; (\text{not } T)) \; T)\\
        &\rightarrow_{\beta}&
        ((\lambda_b . \text{not } T) \; T)\\
        &\rightarrow_{\beta}&
        (((\lambda_b . F) \; T)\\
        &\rightarrow_{\beta}&
        F\\ 
    \end{eqnarray*}
    Lo cual es cierto.
\end{enumerate}
