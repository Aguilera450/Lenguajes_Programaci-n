\textbf{\textcolor{blue}{4.}}\Large
Realiza la representación de los booleanos en el cálculo
$\lambda$ según la representación de los Numerales de Church.
\begin{enumerate}[a)]
%%%%%%%%%%%%%%%%%%%%%%%%      Inciso a    %%%%%%%%%%%%%%%%%%%%%%%%%%%%
    \item Define la función disyunción $\leftrightarrow$ (equivalencia) sobre los boolenos.\\
    Sabemos que a partir de las leyes de equivalencia de la lógica proposicional, tenemos que:
    \begin{equation*}
        a \leftrightarrow b \equiv (a \rightarrow b) \land (b \rightarrow a)\quad \quad \quad \quad
            a \rightarrow b \equiv \neg a \lor b \quad \quad \quad
            b \rightarrow a \equiv \neg b \lor a
        \end{equation*}
        Por lo que finalmente tenemos que:
        $a \leftrightarrow b \equiv (\neg a \lor b) \land (\neg b \lor a )$.\\

        -- Como en clase se vio que $and$ queda definido de la siguiente forma:\\
        $\land \; =_{def} \lambda a. \lambda b.((ab)F)$.\\
        Definimos las siguientes funciones como:\\
        $\lor \; =_{def} \lambda a.\lambda b. ((aT)b)$.\\
        $\neg \; =_{def} \lambda a.aFT$.\\
        $\rightarrow \; =_{def}(\lambda a.\lambda b.\lor (\neg a)b)$.\\
        $\leftrightarrow \;
        =_{def} (\lambda a. \lambda b. \land(\rightarrow ab)(\rightarrow ba))$
%%%%%%%%%%%%%%%%%%%%%%%%      Inciso b    %%%%%%%%%%%%%%%%%%%%%%%%%%%%
    \item Define la función $xor$ (disyunción exclusiva) sobre los booleanos.
    \begin{equation*}
            xor =_{def} \lambda a. \lambda b.(a \; (bFT) \; (bTF))
        \end{equation*}
\end{enumerate}
