\newpage\textbf{\textcolor{blue}{2.}} \Large
Define la función recursiva ocurrencias que recibe dos listas y devuelve una lista de parejas,
en donde cada pareja contiene en su parte izquierda un elemento de la segunda lista y en su
parte derecha el número de veces que aparece dicho elemento en la primera lista. Por ejemplo:

\begin{lstlisting}
>(ocurrencias '(2 6 8 6 2 1 2 2 0 3) '(2 6 9))
' ((2 . 4) (6 . 1) (9 . 0))
\end{lstlisting}

\textbf{Solución.} Para este ejercicio, damos la función recursiva que resuelve
el problema dado y escrita en el lenguaje \code{Racket}. Esto es

\lstinputlisting[language=Scheme, firstline=1, lastline=14]{Script.rkt}
