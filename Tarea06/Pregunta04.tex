\Large\textbf{\textcolor{blue}{4.}}
Observa la siguiente función del lenguaje de programación \code{Racket}.

\begin{lstlisting}
(let ([fib (lambda (n) (if (or (zero? n) (= n 1)) 1 (+ (fib (- n 1)) (fib (- n 2)))))])
      (fib 3))
\end{lstlisting}

\begin{enumerate}[a.]
%%%%%%%%%%%%%%%%%%%%%%%%%%%%%      Inciso A        %%%%%%%%%%%%%%%%%%%%%%%%%%%%%%%%%
\item Prueba la expresión en el intérprete de \code{Racket} y con base en la respuesta 
obtenida, explica el proceso que siguió el intérprete para llegar a ésta. Anexa una 
captura de pantalla del intérprete de \code{Racket} al probar la expresión.
%%%%%%%%%%%%%%%%%%%%%%%%%%%%%      Inciso B        %%%%%%%%%%%%%%%%%%%%%%%%%%%%%%%%%
\item Modifica la función usando el Combinador de Punto Fijo Y .Prueba la expresión en 
el intérprete de \code{Racket} y con base en la respuesta obtenida, explica el proceso que 
siguió el intérprete para llegar a ésta. Anexa una captura de pantalla del intérprete 
de \code{Racket} al probar la expresión.
%%%%%%%%%%%%%%%%%%%%%%%%%%%%%      Inciso C        %%%%%%%%%%%%%%%%%%%%%%%%%%%%%%%%%
\item Modifica la función usando el Combinador de Punto Fijo Z. Prueba la expresión en 
el intérprete de \code{Racket} y con base en la respuesta obtenida, explica el proceso que 
siguió el intérprete para llegar a ésta. Anexa una captura de pantalla del intérprete 
de \code{Racket} al probar la expresión.
\end{enumerate}