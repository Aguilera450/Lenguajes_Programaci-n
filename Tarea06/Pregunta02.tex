\Large\textbf{\textcolor{blue}{2.}}
Explica con tus propias palabras el funcionamiento de las primitivas \code{call/cc} y
\code{let/cc} del lenguaje de programación \code{Racket} y da un ejemplo de uso de cada una.\\

$\rhd$ \textbf{Solución:} El uso de \code{call/cc} y \code{let/cc} se reducen al mismo fin.
No existe diferencia en el resultado al implementar estas funciones, sin embargo existen ciertas
diferencias sutiles, como lo son
\begin{enumerate}
\item \textit{Sintaxis.} Al usar \code{call/cc} tenemos que hacer uso de una \code{lambda} explícita,
mientras que con \code{let/cc} no es necesario.
\item \textit{Semántica.} En cuanto a semántica la diferencia es que con \code{let/cc} la continuación
esta ligada a nuestra variable \code{k} (ya sea explícita o implícita). Al usar \code{call/cc} una
\code{k} que puede ser implícita o explícita, pero funge como la continuación (a diferencia de \code{let/cc}).
\end{enumerate}
A continuación se da una función escrita con \code{call/cc} y \code{let/cc}:
\begin{itemize}
\item Función con base en la primitiva \code{let/cc}:
\begin{lstlisting}
  (define (f n)
     (* 10
        (/ 5
           (let/cc k
                   (k (- 2
                      (+ 1 n)))))))
\end{lstlisting}
\newpage
\item Función con base en la primitiva \code{call/cc}:
\begin{lstlisting}
  (define (f n)
     (* 10
        (/ 5
           (call/cc
                   (lambda(k)
                        (k (- 2
                           (+ 1 n))))))))
\end{lstlisting}
\end{itemize}
En ambos casos se hace uso del ``contexto'' de la función, por lo que sus resultados al ser usadas
serán el mismo.
\hfill $\lhd$
\newpage
