\vspace*{0.3cm}
%%%%%%%%%%%%%%%%%%%%%%%%%%%%% Inciso 2A:
$a$) $e\: =\: \{+\: a\:\; \{+\: b\:\; \{-\: 32\:\; 57\: \}\: \}\: \}$ \newline
\hspace*{0.5cm} \code{(subst (parse e) '$a$ (add (num 3) (num 4)))}   \newline

\hspace*{0.3cm} Primero demos la representación abstracta de la expresión
anterior, esta es
\begin{center}
  \begin{forest}
    [$+$ [$a$] [$+$ [$b$] [$-$ [$32$] [$57$]]]]
  \end{forest}
\end{center}

\hspace*{0.3cm} Luego, apliquemos la instrucción \code{parse} a la expresión
\code{e}, esto es
\begin{eqnarray*}
  \text{\code{(parse e)}} &=& \code{(add (parse($a$)) (parse($\{+\; b\; \{-\; 35\;\; 57\}\}$)))}\\
  &=& \code{(add (id '$a$) (add (parse($b$)) (parse($\{-\; 35\;\; 57\}$))))}\\
  &=& \code{(add (id '$a$) (add (id '$b$) (sub (parse($35$)) (parse($57$)))))}\\
  &=& \code{(add (id '$a$) (add (id '$b$) (sub (num $35$) (num $57$))))}\\
  &=& \code{e'}
\end{eqnarray*}
ahora, apliquemos la instrucción \code{subst} a \code{e'} y los valores indicados,
esto es

\begin{center}
  \fbox{
    \begin{minipage}[b][1\height]%
      [t]{0.867\textwidth}
      \textbf{Obs.} Por simplicidad, asumimos que
      \begin{center}
        \code{subst(e') = (subst (e') 'a (add (num 3) (num 4)))}
      \end{center}
  \end{minipage}}
\end{center}
\begin{eqnarray*}
  \code{subst (e')} &=& \code{(add (subst (id '$a$))
    (subst ((add (id '$b$) (sub (num $35$) (num $57$))))))}\\
  &=& \code{(add (add (num 3) (num 4))
    ((add (subst (id '$b$)) (subst (sub (num $35$) (num $57$))))))}\\
  &=& \code{(add (add (num 3) (num 4))
    ((add  (id '$b$)  (sub (subst (num $35$)) (subst (num $57$))))))}\\
    &=& \code{(add (add (num 3) (num 4))
    ((add  (id '$b$)  (sub (num $35$) (num $57$)))))}.
\end{eqnarray*}

