\vspace*{0.5cm}
%%%%%%%%%%%%%%%%%%%%%%%%%%%%%%% Ejercicio 02:
c{\textcolor{blue}{3.}} Sea la siguiente expresión del lenguaje \code{WAE}.
Da la sintaxis abstracta de la misma, muestra el proceso de evaluación mediante
la función \textbf{interp} y responde las siguientes preguntas.
\newline

$\{\; \code{with}\; \{\; w\;\; 3\; \} $                        \newline
\hspace*{0.6cm} $\{\; \code{with} \{\; x\;\; 9\; \}$           \newline
\hspace*{1.2cm} $\{\; \code{with}\; \{\; y\;\; 4\; \}$         \newline
\hspace*{1.8cm} $\{\; \code{with}\; \{\; z\;\; 11\; \}$        \newline
\hspace*{2.4cm} $\{+\; w\; \{+\; x\; \{+\; y\;\; z\; \}\; \}\; \}\; \}\; \}\; \}\; \}$
\newline

\hspace*{0.3cm} Veamos que la representación de la sintaxis abstracta de la
expresión anterior es
\begin{center}
  \begin{forest}
    [\code{with} [$w$ [$3$]] [\code{with} [$x$ [$9$]]
        [\code{with} [$y$ [$4$]] [\code{with} [$z$ [$11$]]
            [$+$ [$w$] [$+$ [$x$] [$+$ [$y$] [$z$]]]]]]]]
  \end{forest}
\end{center}
Aplicando la función \code{interp} tenemos que
\begin{lstlisting}
> {with {w 3}
        {with {x 9}
            {with {y 4}
                {with {z 11}
                    {+ w {+ x {+ y z}}}}}}}
= {with {x 9}
        {with {y 4}
              {with {z 11}
                    {add (num 3) {+ x {+ y z}}}}}}
= {with {y 4}
        {with {z 11}
              {add (num 3) {add (num 9) {+ y z}}}}}
= {with {z 11}
        {add (num 3) {add (num 9) {add (num 4) z}}}}
= {add (num 3) {add (num 9) {add (num 4) (num 11)}}}
= {add (num 3) {add (num 9) (num 15)}}
= {add (num 3) (num 24)}
= (num 27)
\end{lstlisting} 

\begin{itemize}
\item[$a$)] ¿Cuántas veces se aplica el algoritmo de sustitución para evaluar
  la expresión?

  Para el \code{with} con \code{id = w} tenemos que el algoritmo de sustitución
  se empieza a aplicar desde su cuerpo, después se aplica al valor (\code{val})
  y cuerpo del \code{with} con \code{id = x} (pues \code{w $\not=$ x}), esto
  pasa también, para el \code{with} con \code{id = y} e \code{id = z} (hasta
  aquí hemos aplicado el algoritmo 7 veces, sin aplicarlo al último cuerpo).
  Eventualmente, llegamos al cuerpo de \code{with} con \code{id = z} donde
  aplicamos el agoritmo de sustitución 6 veces más.

  A lo anterior le tenemos que sumar las aplicaciones del algoritmo de sustitución
  si comenzamos desde el \code{with} con \code{id = x}, que son $5$ veces más
  las $6$ aplicaciones del último cuerpo.

  También sumamos las aplicaciones si iniciamos desde el \code{with} con \code{id = y},
  que son $3$ más las $6$ del último cuerpo.

  Por último contamos las aplicaciones con el \code{with} de \code{id = z}, estas son
  $1$ más las $6$ dentro de su cuerpo.
  
  Así, tenemos que en total aplicamos $40$ veces el algoritmo de sustitución (esto
  tomando en cuenta cada llamada recursiva). El algoritmo se aplica para $4$ variables
  en total (que están en el último cuerpo).
   
\item[$b$)] ¿Qué pasaría si añadimos, $100,000$ sumas con $100, 000$
  identificadores nuevos a la expresión? ¿Cuántas sustituciones se
  tendrían que hacer? ¿Qué nos dice esto con respecto al rendimiento
  de la función \textbf{interp}?

  Recordemos que el algoritmo de sustitución es de orden $\mathcal{O}(n^2)$,
  así la cota será de aproximadamente $(100,004)^2$ aplicaciones en el algoritmo
  de sustitución.

  La función \code{interp} es ineficiente para cantidades de identificadores grandes.
\end{itemize}
