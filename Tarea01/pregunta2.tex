\newpage
\section*{Pregunta 2}

LISP es considerado el primer lenguaje de programación funcional. Y esta basado en el Cálculo $\lambda$ que fue desarrollado por Alonzo Church.

\begin{itemize}
\item Investiga y explica brevemente qué elementos del Cálculo $\lambda$ están presentes en LISP e indica por qué crees que pueden usarse en un lenguaje de programación.

$\rhd$ Estos elementos\footnote{Cálculo $\lambda$ y LISP.} realizan
procedimientos principalmente con funciones, aunque una notable diferencia
sería que en el cálculo $\lambda$ no se trabaja con algún tipo de datos,
pues estos no existen, y en LISP tenemos tipos de datos hetereogéneos como
los atómos. Los atómos podrían biyectarse con la aplicación a términos de
una función lambda (sin embargo no son exactamente igual).

Una característica que puede ser común entre el cálculo $\lambda$ y LISP es
la recursión, pues mientras en LISP se utiliza principlamente la recursión
para operar sus funciones, en LISP se utilizan las $\beta$-reducciones

\hfill $\lhd$
\item Menciona cuáles de estos elementos están presentes en el lenguaje de programación JAVA. ¿Acaso estos elementos estaban en las primeras versiones del lenguaje? De no ser así ¿porqué crees que fueron añadidos?
\end{itemize}
