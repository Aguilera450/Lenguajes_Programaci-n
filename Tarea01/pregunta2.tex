\newpage
\section*{Pregunta 2}

LISP es considerado el primer lenguaje de programación funcional. Y esta basado en el Cálculo-$\lambda$ que fue desarrollado por Alonzo Church.

\begin{enumerate}
%---------------------------------------------------- Inciso a)
\item[$a$)] Investiga y explica brevemente qué elementos del Cálculo-$\lambda$
están presentes en LISP e indica por qué crees que pueden usarse en un
lenguaje de programación.

$\rhd$ Estos elementos\footnote{Cálculo-$\lambda$ y LISP.} realizan
procedimientos principalmente con funciones, aunque una notable diferencia
sería que en el cálculo-$\lambda$ no se trabaja con algún tipo de datos,
pues estos no existen, y en LISP tenemos tipos de datos hetereogéneos como
los atómos. Los atómos podrían biyectarse con la aplicación a términos de
una función lambda (sin embargo no son exactamente igual), por ejemplo
\[
(\lambda_x \cdot x)\; a = a
\]
puede equivaler al atómo $a$ en LISP.

Una característica que puede ser común entre el cálculo-$\lambda$ y LISP es
la recursión, pues mientras en LISP se utiliza principlamente la recursión
para operar sus funciones, en el cálculo-$\lambda$ se utilizan las $\beta$-reducciones
(estas muchas veces son recursivas con algunas reglas ya definidas).

La característica de usar paréntesis en su sintáxis es común, sin embargo, esta
no se ve reflejada en su semántica. Pues en LISP representan listas y en el
cálculo-$\lambda$ representa un orden jerárquico en el que se aplican las funciones
base.

En ambos modelos de computo se pueden tener funciones sin aridad definida\footnote{Multiples
parámetros.}. También se sabe que LISP trabaja propiamente con representaciones $\lambda$'s.

\hfill $\lhd$
%---------------------------------------------------- Inciso b)
\item[$b$)] Menciona cuáles de estos elementos están presentes en el lenguaje de programación JAVA.
¿Acaso estos elementos estaban en las primeras versiones del lenguaje? De no ser así ¿porqué
crees que fueron añadidos?

$\rhd$ En JAVA como en muchos otros lenguajes, podemos realizar recursión (tal vez no siempre sea
lo más eficiente).

A partir de la versión $8$ de JAVA se añadieron $\lambda$'s al lenguaje, mencionaremos $2$
razones importantes para que esto haya pasado.
\begin{itemize}
\item Para que más personas se interesen en el lenguaje. El conjunto de personas que
programan tal vez no es el más grande comparado con la humanidad, sin embargo, entre
este subconjunto de personas existen otros tantos, por ejemplo, hay personas que sólo
programan de manera funcional, algunos solo hacen orientación a objetos, y otros
pocos son fan de la programación lógica (y luego están los otros tipos de lengujes
dentro de ambos paradigmas). La intersección no es vacía, pero el que JAVA tenga
$\lambda$'s hace que el lenguaje sea más atractivos para quienes gustan de manejarlas.

\item Facilitan el uso de funciones cuando no hay necesidad de usarlas múltiples veces
y su cuerpo es pequeño. Normalmente se usan como funciones anónimas (aunque son equivalentes,
no son necesariamente iguales a las $\lambda$'s en cuanto a poder computacional).
\end{itemize}

\hfill $\lhd$
\end{enumerate}
