\newpage
\section*{Pregunta 6}
\Large 
A partir de la siguiente función, crea una firma indicando el tipo de entrada que recibirá la función y el tipo de salida que se obtendrá de la función. Asigna un nombre mnemotécnico (es decir que se autodescriba) para la misma. Justifica tu respuesta.

\large
\begin{verbatim}
(define (foo a b)
(cond
[(> (car a) (car b)) (sub1 (foo (cdr a) (cdr b)))]
[else 0]))
\end{verbatim}
%%%%%%%%%%%%%%%%%%%%%%%%%%%%%%%%%%%%%%%%%%%%%%%%%%%
\large
La funcion recibe dos pairs (a y b) y recursivamente va comparando entre la cabezas de ambos pairs, hasta obtener que a $\leq$ b, reduciendo el indice de los elementos del pair (cola) hasta que este sea negativo. De lo contrario arrojara 0 por automático.

\large
\begin{verbatim}
	;; Funcion que recibe dos pairs y regresa su menor indice.  
	;; menor-indice-del-elemento-pequeño: (pair a) (pair b) -> number
\end{verbatim}