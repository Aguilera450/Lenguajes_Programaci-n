\section*{Pregunta 1}
\Large 
Elige 4 lenguajes de programación (uno por cada paradigma), e indica para cada uno de ellos el año de creación, paradigma al que pertenece y principales características.\\
\newline
\large
Aqui pueden comenzar a poner sus respuestas.
\newline
\begin{itemize}
	\item Java. (Orientado a Objetos).\\
	Se creó como una herramienta de programación para ser usada en un proyecto de set-top-box en una pequeña operación denominada the Green Project en Sun Microsystems en 1991. Por un equipo compuesto por trece personas y dirigido por James Gosling. El lenguaje se denominó inicialmente Oak (por un roble que había fuera de la oficina de Gosling), posteriormente cambiado por Green por problemas legales, y finalmente se le renombró Java.\\
	Caracteristicas:
	\begin{itemize}
		\item Es simple: Ofrece la funcionalidad de un lenguaje potente como C y C++.
		\item Es distribuido: proporciona una gran bibliotecas y herramientas para que los programas puedan ser distribuidos.
		\item Portable: Ya que los programas, pueden ejecutarse en cualquier tipo de hardware.
		\item Recolector de basura: Cuando no hay referencias localizadas a un objeto, el recolector se encarga de borrar dicho objeto, liberando así la memoria que ocupaba. Previniendo posibles fugas en la memoria.
		\item Seguro y sólido: Es una plataforma segura para desarrollar y ejecutar aplicaciones que, administra automáticamente la memoria, provee canales de comunicación protegiendo la privacidad de los datos y, al tener una sintaxis compleja evita el quiebre del código, es decir, no permite la corrupción del mismo.
		\item Multihilo: Permite llevar a cabo varias tareas simultáneamente dentro del mismo programa.
		\item Orientado a Objetos: Permite diseñar el software de forma que los distintos tipos de datos que se usen estén unidos a sus operaciones.
	\end{itemize}
	%%%%%%%%%%%%%%%%%%%%%%%%%%%%%%%%%%%%%%%%%%%%%%%%%%%%%%%%%%%%%%%%%%%%%%%%%%
	\item Haskell (Funcional).\\
	La primera versión de Haskell se definió en 1990.\\
	Caracteristicas:
	\begin{itemize}
		\item Puro: Toda operación computacional se contempla como operaciones elementales aritmeticas.
		\item Declarativo: Como su elemento centrales son las funciones, este se centra en determinar qué hace el programa en vez de cómo lo hace.
		\item Polimórfico: Maneja dos tipos de polimorfismo Polimorfismo Paramétrico y Ad-hoc
		\item Estáticamente tipado: Puede encontrar errores antes de que se ejecute el programa ya que los tipos se verifican en tiempo de compilación.
		\item Perezoso: Una expresión no es evaluada cuando se le asocian valore a sus variables, si no que se ejecuta cuando su resultado sea requerido por otras operaciones.
		\item Programas concisos: Usa indentación como medio de estructuración del código.
		\item Sistema de tipos: Tiene un sistema de tipo que requiere un poco de información del programador con los cuales puede detectar una gran variedad de errores de incompatibilidad.
		\item Efectos Monádicos: Para una función dada una misma entrada, siempre se producirá la misma salida, independiemtemente del contexto de las variables.
		\item Concurrente: Permite que, durante un periodo de tiempo, más de un proceso se ejecute. Esto dado a su compilador que permite la copmutación en paralelo.
	\end{itemize}
	%%%%%%%%%%%%%%%%%%%%%%%%%%%%%%%%%%%%%%%%%%%%%%%%%%%%%%%%%%%%%%%%%%%%%%%%%%
	\item Prolog (Lógico).\\
	Fue ideado a principios de los años 70 en la Universidad de Aix-Marseille por Alain Colmerauer y Philippe Roussel. Dando lugar a una versión preliminar del lenguaje Prolog a finales de 1971 y apareciendo la versión definitiva en 1972.
	Caracteristicas:
	\begin{itemize}
		\item Basado en lógica y programación declarativa.
		\item Se centra en la resolución del problema, más que en cómo llegar a esa solución.
		\item A diferencia de otros lenguajes una variable sólo puede tener un valor mientras se cumple el objetivo.
		\item Solo continúa su ejecución, si los objetivos se van cumpliendo.
		\item El usuario se centra más en los conocimientos que en los algoritmos.
		\item Se parte de lo conocido a lo desconocido.
	\end{itemize}
	%%%%%%%%%%%%%%%%%%%%%%%%%%%%%%%%%%%%%%%%%%%%%%%%%%%%%%%%%%%%%%%%%%%%%%%%%%
	\item Fortran (Estructurado).\\
	Desarrollado originalmente por IBM en 1957 para el equipo IBM 704, y usado para aplicaciones científicas y de ingeniería.\\
	Caracteristicas:
	\begin{itemize}
		\item Facilidad con que permite expresar una ecuación.
		\item Potencia en los cálculos matemáticos.
		\item Manejo de archivos.
		\item Formato libre en el código fuente.
		\item Limitado en las aplicaciones de gestión.
		\item Apuntadores y asignación dinámica: Es posible usar almacenamiento dinámico, con lo que se puede hacer que todos los arreglos "trabajen" no importando su tamaño.
		\item Tipos de datos definidos por el usuario: Se pueden definir sus propios tipos compuestos de datos, de forma parecida a como se hace en C o en Pascal.
		\item Módulos: Permiten hacer una programación en un estilo orientado a objetos parecido a como se hace en C++. También son usados para ocultar variables globales.
		\item Sobrecarga de operadores.
	\end{itemize}
	
\end{itemize}
