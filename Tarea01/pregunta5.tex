\newpage
\section*{Pregunta 5}
Investiga que significan los siguientes conceptos en un lenguaje de programación.
Y elabora un pequeño ejemplo ocupando como base al lenguaje de programación HASKELL.
\begin{itemize}
\item Sintaxis.

Se define como el conjunto de reglas que deben seguirse al escribir el código fuente
de los programas para considerarse como correctos para ese lenguaje de programación.

En HASKELL, los nombres de las funciones sólo pueden contener caracteres normales,
es decir, letras, dígitos, comillas y subrayados. El primer carácter de un identificador
de función no puede ser una letra mayúscula ni un dígito. Normalmente se emplea la
notación prefija, sin embargo, es posible usar notación infija al usar acentos franceses,
ejemplo de esto es 8 `div` 3.
\item Semántica.

Descripción de los comportamientos que resultan de la ejecución de un programa o pieza de
software en particular.

En HASKELL, se tiene una evaluación perezosa que permite dar significados o valores
a las funciones hasta que así se requiera.
\item Idioms (Convenciones de programación).

Son costumbres en la programación que son usadas por un grupo (comunidad) de desarrolladores
con frecuencia.

En HASKELL, tenemos que normalmente se usan x:xs para representar una lista, empleamos
las letras $a, b, c$ para valores numéricos (también $x, y$ o $n, m, k$). Se suela
usar $x' = x + y$ (donde basta con que $y$ sea del tipo de $x$), etc.
\item Bibliotecas.

Son un conjunto de herramientas ya construidas y que muchas veces trae por omisión
el lenguaje, en otros casos hay que darle la referencia a estas (cargarlos).

En HASKELL, tenemos algunas como Math, Eq, etc.
\end{itemize}
