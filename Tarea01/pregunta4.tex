\newpage
\section*{Pregunta 4}
\Large 
¿Cuál de los paradigmas de lenguajes de programación, es el más adecuado para resolver los siguientes problemas? Justifica en cada caso.\\\\
%%%%%%%%%%%%%%%%%%%%%%%%%%%%%%%%%%%%%%%%%%%%%%%%%%%%%%%%%
a) Se requiere desarrollar un sistema que simule un modelo de sociedad de organización de termitas. Este sistema se compone, de manera general de: un espacio que las termitas recorrerán y en el cuál se encuentran astillas esparcidas, las termitas siguen las siguientes reglas de comportamiento.\\
\begin{itemize}
\item Caminan aleatoriamente hasta encontrar una astilla.
\item Si la termita se encuentra cargando una astilla, la suelta y continúa caminando aleatoriamente.
\item Si la termita no está cargando una astilla la toma y continúa caminado aleatoriamente con la astilla
\end{itemize}
\large

Debido a las especificaciones del problema, determinamos que  para este caso, el paradigma que más nos conviene usar
para resolver el problema es el paradigma de programación orientada a objetos por las siguientes razones:

\begin{itemize}
    \item Se menciona en el primer comportamiento que las termitas caminan aleatoriamente hasta encontrar una astilla,
    por lo se puede entender que hay varias termitas a la vez en el modelo, entonces sería mucho más sencillo implementar
    objetos e instancias de las termitas.

    \item Se está describiendo como tal el comportamiento de las termitas, por lo que se puede controlar su acciones por
    medio de iteraciones mucho más facil que con recursión.
\end{itemize}

